\documentclass[letterpaper, 11pt]{article}
\usepackage{enumitem, titlesec}
\usepackage[colorlinks=true, urlcolor=black]{hyperref}
\usepackage[margin=0.5in]{geometry}

% disable page number
\pagestyle{empty}
% section setting
\titleformat{\section}{\scshape\Large}{}{0em}{}[\titlerule]
% vsapce between subsection
\newcommand{\subsectionvspace}{\vspace{6pt}}
% line space of item same to default line space of the body paragraph
\setlist[itemize]{noitemsep, topsep=0pt}
% no automatic indentation at the beginning of the body paragraph
\setlength{\parindent}{0pt}


\begin{document}


\begin{center}
    \textbf{\LARGE Tianrui (Eric) Qi} \\
    +1(518)-961-3370 
    $\cdot$ 
    tianrui.qi@gatech.edu
    \\
    Linkedin: \href{https://www.linkedin.com/in/tianrui-qi/}{\underline{tianrui-qi}}
    $\cdot$
    GitHub: \href{https://github.com/tianrui-qi}{\underline{tianrui-qi}}
\end{center}


\section{Education}


    \textbf{Bachelor of Science in Computer Science} \hfill 
    \textbf{Jan 2023 - (May 2025)} \\
    \textit{Georgia Institute of Technology, Atlanta, GA} \\ 
    \textit{Dean's Honor List all semesters, GPA: 3.92/4.00}
    \begin{itemize}
        \item Minor in Physics.
    \end{itemize}

    \subsectionvspace

    \textbf{Bachelor of Science in Computer Science; Double Major in Mathematics} \hfill 
    \textbf{Sep 2020 - Dec 2022} \\
    \textit{Rensselaer Polytechnic Institute, Troy, NY} \\ 
    \textit{Dean's Honor List all semesters, GPA: 3.73/4.00}
    \begin{itemize}
        \item Minor in Economics.
    \end{itemize}


\section{Experience}


    \textbf{Startup Founder} \hfill 
    \textbf{Aug 2024 - Present} \\
    \textit{Georgia Institute of Technology, Atlanta, GA} \\
    \textit{CREATE-X Idea to Prototype, Mentor: Dr. Xuanwen Hua}
    \begin{itemize}
        \item Conceptualizing an AR platform that simulate interactions with 2D surfaces in a 3D space.
        \item Exploring Apple's AR platforms and ARKit and gathering user feedback to identify potential applications. 
        \item Developing an iOS app that transforms a 3D space into 2D canvas for creation and projects back for viewing.
    \end{itemize}

    \subsectionvspace

    \textbf{Undergraduate Research Assistant} \hfill 
    \textbf{Apr 2023 - Present} \\
    \textit{Georgia Institute of Technology, Atlanta, GA} \\
    \textit{Jia Laboratory for Systems Biophotonics, Principal Investigator: Dr. Shu Jia}
    \begin{itemize}
        \item Engineered a scalable 3D U-Net pipeline based entirely on simulated data for volumetric localization in single-molecule super-resolution microscopy, achieving resolution down to 20 nm.
        \item Developed a patch-based prediction pipeline that flexibly adapts to various input volume size and achieves a 100x speedup over conventional deterministic localization methods.
        \item Integrated the redundant cross-correlation algorithm for drift correction with the deep learning-based prediction pipeline.
    \end{itemize}

    \subsectionvspace

    \textbf{Co-op} \hfill 
    \textbf{Jan 2024 - Aug 2024} \\
    \textit{Regeneron Genetics Center, Tarrytown, NY} \\
    \textit{Therapeutic Area Genetics, Manager: Dr. Jing He}
    \begin{itemize}
        \item Obtained a vector representation for each whole exome sequencing (WXS) sample by creating a bio-meaningful space using BERT-based large language models (LLMs) and unsupervised feature selection.
        \item Demonstrated that the representations capture sample-wise differences by predicting immune system indicators of The Cancer Genome Atlas Program (TCGA) skin cancer samples.
        \item Scaled up the pipeline to handle 1,000 WXS samples with 100 billion DNA sequences by optimizing parallel computing for high-performance computing (HPC) and enhancing file system efficiency through hashing.
    \end{itemize}

    \subsectionvspace

    \textbf{Undergraduate Research Assistant} \hfill 
    \textbf{Nov 2021 - Dec 2022} \\
    \textit{Rensselaer Polytechnic Institute, Troy, NY} \\
    \textit{AI-based X-ray Imaging System Lab, Principal Investigator: Dr. Ge Wang}
    \begin{itemize}
        \item Derived backward propagation formulation for quadratic neural networks and compared forward and backward propagation between quadratic and conventional neural networks mathematically.
        \item Implemented forward propagation, backward propagation, and training process of quadratic and conventional neural networks explicitly using NumPy in Python.
        \item Demonstrated that single-layer quadratic neural networks rivals conventional neural networks with hundreds of neurons in classifying simulated and real-world Gaussian mixture data.
    \end{itemize}

    \subsectionvspace

    \textbf{Undergraduate Teaching Assistant} \hfill 
    \textbf{Sep 2022 - Dec 2022} \\
    \textit{Rensselaer Polytechnic Institute, Troy, NY} \\
    \textit{Foundations of Computer Science, Instructor: Dr. David Goldschmidt}
    \begin{itemize}
        \item Led weekly recitation sessions to help students understand course material.
        \item Assisted students' understanding of weekly lab exercises and graded assignments and exams.
    \end{itemize}


\section{Publications}


    Keyi Han$^\dag$, Xuanwen Hua$^\dag$, \textbf{Tianrui Qi}$^\dag$, Zijun Gao, Xiaopeng Wang, Shu Jia, ``Volumetric Reconstruction and Localization Networks for 3D Single-molecule Localization Microscopy,'' \textit{manuscript in preparation} (expected 2024).

    \subsectionvspace

    \textbf{Tianrui Qi}, Ge Wang, ``Superiority of quadratic over conventional neural networks for classification of gaussian mixture data,'' \textit{Visual Computing for Industry, Biomedicine, and Art} (2022).

    \subsectionvspace

    \textit{$\dag$ denotes co-first authors}


\section{Projects}


    \textbf{Alternating Direction Method of Multipliers for Support Vector Machine} \hfill 
    \textbf{Jan 2022 - May 2022} \\
    \textit{Rensselaer Polytechnic Institute, Troy, NY} \\
    \textit{Computational Optimization, Instructor: Dr. Yangyang Xu}
    \begin{itemize}
        \item Formulated the primal and augmented dual optimization problems for support vector machine (SVM) objective and developed alternating direction method of multipliers (ADMM) solver.
        \item Implemented the ADMM solver in MATLAB and reported the primal and dual feasibility violation at each outer iteration for the testing datasets.
    \end{itemize}

    \subsectionvspace

    \textbf{Windows of Susceptibility Analysis for Brain Diseases} \hfill 
    \textbf{Jan 2022 - Mar 2022} \\
    \textit{Rensselaer Polytechnic Institute, Troy, NY} \\
    \textit{Data Mathematics, Instructor: Dr. Kristin Bennett}
    \begin{itemize}
        \item Performed the windows of susceptibility analysis based on mouse data from a similar brain-in-a-dish model for mice using R with k-means clustering and principal component analysis (PCA).
        \item Analyzed the same sets of microcephaly-associated genes and Zika-associated genes and detected similar windows of susceptibility for Microcephaly and Zika-induced microcephaly in mice as in humans.
    \end{itemize}

    \subsectionvspace

    \textbf{MIPS Processor in C} \hfill 
    \textbf{Sep 2021 - Dec 2021} \\
    \textit{Rensselaer Polytechnic Institute, Troy, NY} \\
    \textit{Computer Organization, Instructor: Dr. Konstantin Kuzmin}
    \begin{itemize}
        \item Designed a datapath for a reduced MIPS instruction set architectures (ISA) that support I-type instructions including \verb|lw|, \verb|sw|, \verb|beq|, \verb|addi|, R-type including \verb|and|, \verb|or|, \verb|add|, \verb|sub|, \verb|slt|, \verb|jr|, and J-type including \verb|j|, \verb|jal|.
        \item Implemented the datapath through a full gate-level circuit in C, including components of the processor like memory, control, arithmetic logic unit (ALU), decoder, adder, multiplexor, etc.
    \end{itemize}


\section{Skills}


    \textbf{Programming Languages:} Python (PyTorch, NumPy, pandas), MATLAB, Java, C, C++, R, Swift (ARKit), Bash, MIPS. \\
    \textbf{Development Tools:} Git, Conda, VSCode, JetBrains (PyCharm, IntelliJ, CLion, Android Studio), RStudio, Xcode. \\
    \textbf{Computing Plantforms:} Linux (Ubuntu), AWS (EC2, S3), HPC (Slurm). \\
    \textbf{Software:} LaTeX, ImageJ, Adobe Illustrator. \\
    \textbf{Laboratory:} optics and laser alignment, fluorescence imaging, fluorescence labeling, cell culture maintenance. \\
    \textbf{Communication:} English (Proficient), Mandarin (Native).


\end{document}
